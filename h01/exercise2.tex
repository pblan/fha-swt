
\documentclass{abgabe}
\begin{document}

\begin{questions}
    \qformat{\thequestion. \textbf{\thequestiontitle} \hfill}
    \titledquestion{Charakterisierung Softwaretechnik}
    In der Vorlesung wurde eine zusammenfassende Definition von Softwaretechnik vorgestellt.
    \begin{parts}
        \part 
        Welche softwaretechnischen Mittel werden in Softwareprojekten eingesetzt?
        \begin{solution}
            Aus der Vorlesung:
            \begin{itemize}
                \item \emph{Werkzeuge}: Effektive Entwicklung:
                      \subitem SEU, Build, Test, Quality, VCS, \ldots 
                \item \emph{Methoden}: Beschreiben bewährte Verfahren:
                      \subitem OOA, OOD, TDD, Coding Conventions, CI/CD, \ldots 
                \item \emph{Vorgehensmodelle}: Standardisierte Prozessbeschreibungen:
                      \subitem Scrum, Kanban, V-Modell, RUP, \ldots 
                \item \emph{Dokumentation}: Einheitliche Notationen für Entwicklungsergebnisse:
                      \subitem UML, Templates, Tracebility im Prozess, \ldots 
            \end{itemize}
        \end{solution}
        
        \part 
        Welche Eigenschaften von Software-Produkten sollen dadurch erreicht werden?
        \begin{solution}
            Aus der Vorlesung: 
            \begin{itemize}
                \item Performance
                \item Effizienz
                \item Wartbarkeit
                \item Zuverlässigkeit
                \item Benutzbarkeit
            \end{itemize}
        \end{solution}
        
        \part 
        Wodurch unterscheidet sich Softwaretechnik damit von handwerklich ordentlichem Programmieren?
        \begin{solution}
            Aus der Vorlesung: 
            
            \begin{displayquote}
                Software-Engineering ist der Einsatz
                \begin{itemize}
                    \item \emph{qualifizierter Methoden}, \emph{Werkzeuge und Vorgehensmodelle} zum \emph{Erstellen und Betreiben von Software} mit dem Ziel,
                    \item einerseits die Softwarekosten bei der \emph{Entwicklung, Wartung und Erweiterung} von Programmsystemen zu senken und
                    \item andererseits eine höhere \emph{Systemqualität} zu erreichen.
                \end{itemize}
            \end{displayquote}
        \end{solution}
    \end{parts}
\end{questions}
\end{document}