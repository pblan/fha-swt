
\documentclass{abgabe}
\begin{document}

\begin{questions}
    \qformat{\thequestion. \textbf{\thequestiontitle} \hfill}
    \titledquestion{Veranstaltungsverwaltungssystem}

    Gegeben sei folgende Beschreibung für ein Veranstaltungsverwaltungssystem:

    Personen haben Zeichenketten als Name.
    Studenten sind Personen, erben also die Eigenschaften von Person, haben aber zusätzlich eine ganzzahlige Matrikelnummer und nehmen an beliebig vielen Veranstaltungen teil.
    Eine Veranstaltung hat potentiell beliebig viele Teilnehmer, wird aber von einem, zwei oder drei Mitarbeitern betreut.
    Eine Veranstaltung hat eine Veranstaltungsnummer und einen Titel.
    Seminare und Vorlesungen sind spezielle Veranstaltungen.
    Ein Seminar hat eine begrenzte Anzahl an Plätzen, für eine Vorlesung wird eine Klausur angeboten oder nicht.
    Mitarbeiter sind Personen und betreuen eine bis fünf Veranstaltungen und haben eine Personalnummer.
    Professoren und Assistenten sind Mitarbeiter.
    Assistenten sind bei genau einem Professor beschäftigt und haben eine bestimmt Finanzierung (Zeichenkette).
    Ein Professor hat ein Lehrgebiet (Zeichenkette), beschäftigt beliebig viele Assistenten und ist Inhaber von genau einem Lehrstuhl.
    Ein Lehrstuhl hat eine Bezeichnung und genau einen Professor als Inhaber.

    Erstellen Sie anhand der obigen Beschreibung ein \emph{Klassendiagramm}.
    Ihr Diagramm sollte folgende Punkte beinhalten:
    \begin{itemize}
        \item \emph{Generalisierungsbeziehungen},
        \item \emph{Assoziationen} mit Assoziationsnamen und Leserichtung,
        \item \emph{Multiplizitäten} sowie
        \item \emph{Attributnamen} und (sinnvolle) \emph{–typen}.
    \end{itemize}

    Finden Sie jeweils ein Beispiel, bei dem eine Aggregations- und eine Kompositionsbeziehung sinnvoll ist.
    Erläutern Sie kurz den Unterschied zwischen Aggretation und Komposition anhand des Bespiels.
    \newpage
    \begin{solution}
        \begin{center}
            \includegraphics[width=\textwidth]{swt_h07_veranstaltungsverwaltung.pdf}
        \end{center}

        Reddit-User \href{https://www.reddit.com/r/javahelp/comments/gpnqij/comment/frnuudx/?utm_source=share&utm_medium=web2x&context=3}{raja\_42} fasst den Unterschied zwischen Aggregation und Komposition gut zusammen:
        \begin{displayquote}
            Association is a relationship between two entities. Kind of, associated with, in English.

            E.g. An Employee class will have a property which is a list of Projects.

            Composition is when a container entity has child entities which cannot survive on their own.

            E.g. A Shopping Cart class is composed of a list of Cart Items. When a Shopping Cart is deleted, the Cart Items cannot exist conceptually. So they are transient in nature.

            Any physical representation of them in database etc. will follow this principle. E.g. There will never be a Cart Item record in the DB without a Shopping Cart record id.

            Finally, an aggregation is like composition except the contained entities can exist on their own.

            E.g. When you are selecting a group of people in a classroom for a science project, you model few entities or classes.

            A Student class for every student in the class. A Professor entity. Etc.

            A ProjectGroup class is an aggregation which then has a list of students, a professor etc.

            Both are independent concepts. Dissolving a project group doesn't make the corresponding students and professors vanish from the system. They still exist.

            In physical form, they represent DB records that don't get deleted when the container record is deleted. This just means that they are independent entities that can live on their own.
        \end{displayquote}
    \end{solution}
\end{questions}
\end{document}