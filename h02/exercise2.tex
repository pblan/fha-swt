
\documentclass{abgabe}
\begin{document}

\begin{questions}
    \qformat{\thequestion. \textbf{\thequestiontitle} \hfill}
    \titledquestion{Veränderte Anforderungen}
    Einem Kunden fällt nach Zweidritteln erfolgreicher Projektlaufzeit ein, dass er eine bereits entwickelte Funktionalität nicht benötigt, dafür aber eine andere wünscht. 
    Beschreiben Sie kurz, wie man bei Nutzung der folgenden Vorgehensmodelle darauf reagieren würde:
    \begin{parts}
        \part 
        Wasserfallmodell
        \begin{solution}
            \begin{center}
                \includegraphics[width=0.5\textwidth]{crying.png}
            \end{center}
            
            Und Erklärung:
            
            Wenn man erst einmal davon absieht, dass das Wasserfall-Modell nach Royce bewiesenermaßen \href{http://valueatwork.se/waterfall-model-probably-the-most-costly-mistake-in-the-world/?lang=en}{ohnehin nicht funktioniert}, sieht das Wasserfallmodell keine Möglichkeit der Reevaluation vor. 
            Ergo wird bei striktem Ablauf nach dem Wasserfallmodell die Anforderung einfach ignoriert.
        \end{solution}
        
        \part 
        Iterative Entwicklung
        \begin{solution}
            Man kann hoffen, dass die zu verändernde Funktionalität nicht zu viele Änderungen am bisher entwickelten System hervorruft. 
            Ist dies der Fall, kann man in der nächsten Iteration die benötigten Änderungen vornehmen.
            
            Sind allerdings zu viele Änderungen nötig, muss potentiell die gesamte Codebase verändert oder neugebaut werden. 
            Das kann entweder einen (quasi) Neustart, oder das gleiche Ergebnis wie in (a) hervorrufen.
        \end{solution}
        
        \part 
        Inkrementelle Entwicklung
        \begin{solution}
            Man kann hoffen, dass das zu verändernde Inkrement nicht zu viele Änderungen am bisher entwickelten System hervorruft. 
            Ist dies der Fall, kann man in dem nächsten Inkrement die benötigten Änderungen vornehmen.
            
            Sind allerdings zu viele Änderungen nötig, muss potentiell die gesamte Codebase verändert oder neugebaut werden. 
            Das kann entweder einen (quasi) Neustart, oder das gleiche Ergebnis wie in (a) hervorrufen.
        \end{solution}
        
        \newpage
        \part 
        Wo würden Sie das SCRUM Modell einordnen?
        \begin{solution}
            Aus \href{https://de.wikipedia.org/wiki/Scrum}{Wikipedia}: 
            \begin{displayquote}
                Der Ansatz von Scrum ist empirisch, inkrementell und iterativ. 
                Er beruht auf der Erfahrung, dass viele Entwicklungsprojekte zu komplex sind, um in einen vollumfänglichen Plan gefasst werden zu können. 
                Ein wesentlicher Teil der Anforderungen und der Lösungsansätze ist zu Beginn unklar. Diese Unklarheit lässt sich beseitigen, indem Zwischenergebnisse geschaffen werden. 
                Anhand dieser Zwischenergebnisse lassen sich die fehlenden Anforderungen und Lösungstechniken effizienter finden als durch eine abstrakte Klärungsphase. 
                In Scrum wird neben dem Produkt auch die Planung iterativ und inkrementell entwickelt. Der langfristige Plan (das Product Backlog) wird kontinuierlich verfeinert und verbessert. 
                Der Detailplan (das Sprint Backlog) wird nur für den jeweils nächsten Zyklus (den Sprint) erstellt. 
                Damit wird die Projektplanung auf das Wesentliche fokussiert.
            \end{displayquote}
        \end{solution}
    \end{parts}
\end{questions}
\end{document}